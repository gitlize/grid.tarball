{\nopagenumbers} % use plain, no LaTeX

\magnification\magstephalf

\font\bbbf = csb10 scaled \magstep3
\font\bbf  = csb10 scaled \magstep1

\def\titul #1\par {\line{\hss\bbbf #1\hss}\medskip}
\def\tit #1\par {\bigskip{\bbf #1}\par\nobreak\medskip\noindent}
\def\para #1 #2 #3\par {\par\smallskip
   {\noindent{\tt #1 #2}\hfill #3\leftskip=0pt \par}
   \nobreak\noindent}
\def\date #1\par {\medskip\line{\hss\bf #1\hss}\medskip}
\def\authors #1\par {\line{\hss\bf #1\hss}\medskip}

\parindent=12pt

\catcode`\"=13
\def"{\leavevmode\hbox\bgroup\let"=\egroup\setverb\tt}
\def\setverb{\def\do##1{\catcode`##1=12}\dospecials\obeyspaces}
\def\begtt{\medskip\bgroup
   \nobreak\setverb \parskip=0pt %\parindent=0pt
   \def\par{\endgraf\penalty500 }
   \catcode`\"=12\catcode`\~=13 \obeylines
   \startverb}
{\catcode`\|=0 \catcode`\\=12
  |gdef|startverb#1\endtt{%
        |tt#1|nobreak|egroup|penalty1000|medskip|scannexttoken}}
{\obeyspaces\gdef {\ }}
\long\def\scannexttoken#1{\ifx#1\par\else\noindent#1\fi}

\def\bod {\par\smallskip \noindent\hbox to\parindent{\hss$\bullet$\hss}}

\catcode`\<=13
\def <#1>{{$\langle$\it#1$\rangle$}}
\def\at{@}

\titul --- grid --- 

\titul The griddler (nonogram) solver
%%%%%%%%%%%%%%%%%%%%%%%%%%%%%%%%%%%%%

\date 27. 9. 2003              

\authors Mirek Ol\v s\'ak + Petr Ol\v s\'ak 
\ ({\tt mirek\at olsak.net, petr\at olsak.net})

This program solves the ``griddler puzzles'' (somebody call it as
``nonogram puzzles''). The rectangle of square puzzles is given and
the lengths of black (or colored) blocks is given around sides of
the rectangle. You have to color the puzzles in such way to all data around
puzzles stays correct. You can get more info from:

\begtt
www.griddlers.net
\endtt

The program "grid" solves the black+white or colored square puzzles.
The triangular puzzles in rectangle grid are possible too. Moreover,
the program can solve ``triddlers'' (triangular grid with hexagonal 
circumference) in black+white and colored version.

The program "grid" gives you the possibility to solve all puzzles
or it can read the partially solved puzzles and give you only 
a hint about next one step. Moreover, program can check your partially
solution and show you your errors in it (but it cannot show you the
whole solution).


\tit How to install
%%%%%%%%%%%%%%%%%%%

You can compile the program from source file by the command:

\begtt
cc -O2 -o grid grid.c
strip grid
\endtt

The option "-O2" is supported by GNU "gcc" (speed optimization).
If your compiler does not support this option then you cannot use it.
The command "strip grid" is optional (may be not implemented on every
platforms). It removes the debug info from binary code.

The source code is based only on standard C library. It means, that
the program will be compiled without problems at 
{\it arbitrary computer platform}.


\tit How to run the program 
%%%%%%%%%%%%%%%%%%%%%%%%%%%

You can run the program by the command:

\begtt
grid file
\endtt
%
where "file" is the name of the input file where the task with block
lengths in rows and columns of a grid is declared. The format of input
file is described below.

A several cases can be occur during solution:

\bod The solution is found.
  In this case, program prints OK and prints the solution on terminal.
  Moreover, it saves the solution as graphical output in "file.xpm".
  You can see and print this solution using Gimp, for example.

\bod There is a conflict in the task.
  Program simple checks the consistency of the task before it starts 
  the solution: it checks if the number of puzzles of each color
  is the same from ``row point of view'' as from ``column point of view''.
  It this is not true, the error message occurs and program terminates.

\bod There is a conflict in the task, but solution was started.
  The unsolved puzzles are printed by question mark. Program prints KO.
  This case occurs only if there is a conflict in the task but
  the simple test (see previous item) did not detect this problem.
  The program points to the row/column where the problem is and terminates.

\bod There exist more than one solution.
  Program prints all solutions and it saves only the first solution to
  XPM graphic file. This behavior can be changed by command line
  options, see below.


\tit Command line options
%%%%%%%%%%%%%%%%%%%%%%%%%

\begtt
grid [options] mainfile
\endtt

\noindent "-help"

\bgroup \leftskip=\parindent \parindent=0pt
\def\bod{\par}

   Program prints a short help about options to the terminal and 
   terminates.

\para -p <number>                default value: "-p 0"

   Program enters to the ``pause mode'' after <number> steps.
   If it is in pause mode then it pauses after each step
   and prints the partial solution on the terminal. The new
   solved puzzles (from the last step) is printed by "#" (color)
   and "-" (background). If the puzzles are colored and "-out 3" 
   is given then program prints all ``color layers'' in pause mode.
   The "-p 0" means that the pause mode will be never reached,
   "-p 1" means that pause mode is active immediately after first step.
   The word ``step of program'' will be explained at the end
   of this chapter.

\para -stop <number>             default value: "-stop 0"

   The meaning is the same as in "-p <number>" option. The only
   difference is that the program does not pauses but terminates.
   If "-stop 0" is given then program terminates after all solutions
   are found or after conflict is found or after "-total" solutions
   are found.

\para -total <number>            default value: "-total 30"

   If the <number> different solutions of one task is found then
   program terminates and it does not find another solutions.
   If "-total 0" is given then all solutions will be found.

\para -xpm <number>              default value: "-xpm 1"

   The first <number> solutions will be saved to "*.xpm" files.
   If "-xpm 1" is given then the program creates the file with the
   same base-name as "mainfile" but appends the ".xpm" extension.
   If "-xpm 2" or more is given then the suffix "<solution number>.xpm"
   is appended to each name of XPM file. It means that more than one
   XPM file can be created. The <solution number> has the same
   number of digits as the <number> given by "-xpm" option. It means
   that left trailing zeros can be appended to <solution number>.
   If "-xpm 0" is given then no XPM output is created.

\para -log <number>              default value: "-log 2"

   The <number> means the verbosity level of the output to stdout.
\bod   "-log 0" \dots\ Only solutions are printed. The output format of
       solutions can be controlled by "-out" option.
\bod   "-log 1" \dots\ Program prints the number of solutions and
       the final statistic about number of steps and successful/all
       calls of various line solvers. 
\bod   "-log 2" \dots\ Moreover, the number of steps are printed
       during solving.
\bod   "-log 3" \dots\ Moreover, the state of solved lines are printed.
\bod   "-log 4" \dots\ Moreover, the internal information from line
       solvers is printed.

\para -out <number>              default value: "-out 2"

   The solution printing format is controlled by "-out" option.
\bod   "-out 0" \dots\ nothing is printed.
\bod   "-out 1" \dots\ solution without numbers of rows/columns around it.
\bod   "-out 2" \dots\ solution including numbers of rows/columns around it.
\bod   "-out 3" \dots\ no-definitive solutions (during pause mode for example)
        are printed in all color layers.

\para -cmp <file>               

   The <file> includes the partially solved puzzles in the same
   format as printed by program if "-out 2" is given. More details
   about this format is described below.
   If "-cmp <file>" is given then program finds the first solution 
   but does not print it and does not save it to XPM. Program only
   checks the differences between solution and partially solved 
   puzzles in <file> and prints (by "#" and~"-" characters) the 
   differences. Question marks are unchanged. This is a ``hint mode''
   because program prints only a hint about bugs in your partially
   solved puzzles.

\para -ini <file>

   Program starts the solution from the state given in partially
   solved puzzles in <file>. In this case program sets 
   "-stop 1" in order to only hint about one next step is printed.
   If you want to run more steps then you have to use
   "-stop" option explicitly after "-ini" option. Example:
   "grid -ini my -stop 0 problem.g"

\para -bl <number>                  default value: "-bl 7"

   First, program examines only rows and columns where less or equal than
   <number> blocks are present. If this limit causes no new solved
   puzzles then program enter the ``full mode'' where all rows and
   columns are examined.

\par\egroup \medskip

You can write options in arbitrary order separated by space.
If you use the same option more than once then the last one 
is significant. If you write "-" character instead of name of file
then standard input is used. It is possible, for example, the
following fitting of input files:

\begtt
cat main.g inifile | grid -xpm 0 -log 0 -out 1 -ini - - > hint-file
\endtt

Now, we describe the word {\bf step of the program}. One step is the
pass through all rows (step type~{\bf r}) or through all columns (step
type~{\bf c}). These two types of steps alternates. The fast (so called
left-right) line solver is used in this type of steps. This line
solver is not able to find all new solved puzzles. This implies that
the steps of type~{\bf r} and~{\bf c} can fail (it means they gives no new solved
puzzles). In such case the step of type~{\bf i} (intensive) is invoked.
The slow but elaborate line solver is used in this step. 
Program is finding only first row or column which gives
new solved puzzles using intensive line solver. Then program 
finishes this type of step and enter the ``normal'' type of steps
{\bf r} or {\bf c}. These normal steps can fail again and then the 
intensive type of step is entered again. If the intensive step
fails then program enters to step type {\bf t} (test). In this case 
program tries to substitute some unsolved puzzle to a definitive
color (or background) and run the normal steps {\bf r} or {\bf c} again.
If conflict occurs in this situation then program tries to substitute
unsolved puzzle by another color and run normal steps again.
And so on, and so on\dots

If triddlers are active then three ``normal'' step types are in progress:
{\bf r}~-- rows, {\bf c}~-- columns from bottom, {\bf e}~-columns from
top of the hexagonal. If these steps fail then the steps type~{\bf i}
and (may be) type~{\bf t} are invoked as in griddlers.


\tit The format of main input file
%%%%%%%%%%%%%%%%%%%%%%%%%%%%%%%%%%

If you have only two colors (black and white) puzzles then the format
of the input file is simple:

\begtt
Arbitrary text in zero or more lines. This text is used for comments
only and it is ignored if no colon, no hash is present as first
character of the line. 
The first character of the whole file cannot be the decimal digit.
: the colon at the first position of the line starts row declaration
... data from rows
    each line represents one row data with block lengths
: the colon at the first position of the line starts column declaration
... data from columns
    each line represents one column data with block lengths    
: the colon at the first position of the line ends the input
The arbitrary text here will be ignored.
\endtt

The rows and columns data includes decimal numbers separated by
space (or more spaces or tabulators). The arbitrary spaces and
tabulators can be before the first number too.
The numbers denotes the lengths of the blocks.
The empty line is essential: it denotes the row/column without
any blocks. The example of this type of input is in the 
file~"kocka.g".

Colored puzzles have the similar format of input file, but the colors
declaration have to be present. An example including detail
description of this format can be found in "oko.g" and "ruze.g" files.

The triangle puzzles (triangles by www.griddles.net) have the same
input file format as the colored puzzles. The example including the 
description of this format can be found in "alladin.g" file.

The triangle puzzles (triangles by the journal ``Malovane krizovky'',
Silentium s.r.o.) have the similar format as colored puzzles, but
you have to declare the left-glue and right-glue triangles by
{\tt\char`\<} or {\tt\char`\>} characters. The example including
detail format description can be found in "brontik.g" file. 

The triddlers are declared by "#T" or "#t" at the first column
before color declaration. The six data groups (separated by
colons) are expected instead two data groups in rectangular puzzles.
The first group means rows from side~A, second rows from side~B,
third columns from side~C, fourth columns from side~D, fifth columns
from side~E and sixth columns from side~F. The sides of the hexagonal are 
labeled in "tkocka.g" file. Simply speaking you begin to read the data
at left upper corner and go counterclockwise around the hexagonal.
Warning: the block lengths of columns, which begin at the bottom of
hexagonal, have to be read {\it from underneath upstairs}.
See the examples in "tkocka.g" and "vcely.g" files.

The program is able to read the black+white puzzles format
used by "mk.exe" program for MS~Windows 
("http://frix.fri.utc.sk/~johny/mk43frm.php").
The input mode for such format is activated automatically if the
first character of the input file is a decimal digit.
See the example in the file "levikral.mk" or in another "*.mk" files.


\tit Format of input file with partial solution
%%%%%%%%%%%%%%%%%%%%%%%%%%%%%%%%%%%%%%%%%%%%%%%

These files are used by "-ini" and "-cmp" options.
The format is compatible with the grid output on the terminal:

\begtt
arbitrary text
:::: four colons say that the following line starts data input
   : the data lines
   : the number of these lines have to be the same as the number of rows
arbitrary text
\endtt

Each data line has the format:

\begtt
<arbitr.text><colon><ignored char><data characters><arbitr.text>
\endtt
%
where the number of <data characters> have to be the same as
the number of columns.

\let\bod=\par

The <data character> is one of the following:

\medskip
\bod question mark or period --- unsolved puzzle
\bod space or minus --- the background color
\bod asterisk or hash mark --- black color
\bod the <outchar> from color declaration --- this color
\medskip

You can create the partial solution file very simply:

\begtt
grid -stop 1 task.g > task.p
\endtt

And you can go on:

\begtt
grid -ini task.p task.g > task2.p
\endtt

If you are solving triddlers then you can use the same file format
only with the following difference. Each data line has the format:
\begtt
<arbitr.text><(back)slash><ignored char><data characters><arbitr.text>
\endtt
%
where <(back)slash> is the normal slash ("/") or the backslash ("\")
character.


\tit The example of usage
%%%%%%%%%%%%%%%%%%%%%%%%%

In~UNIX shell:

\begtt
for i in *.g *.mk; do grid -out 0 $i; done
gimp *.xpm
\endtt


\tit The insides
%%%%%%%%%%%%%%%%%

of the program is described in source "grid.c" in detail.
Of course, {\it very detailed} description is here -- the amount of
comments are great than the amount of code like in "tex.web" source.

We assume that the usage of this program without usage of your own
head brings no enjoyment. More enjoyment occurs if you are solving
these puzzles manually. Most enjoyment occurs if you are studying of the
puzzle solvers principles and of possibility of implementation these
principles into the computer program.
You can do this, it is sufficient to use some text editor, open
the "grid.c" source and read\dots

Sorry, the comments in "grid.c" are only in our mother tongue.
It means Czech, no English.

We spended many hours of time optimization of our program. We
rejected the {\it brutal force\/} method and used the 
{\it brutal intelligence\/} method. We assume that our program
belogs to the fastest programs in its category and to the
best documented programs. 

The another advantage of this program is its
independence of the computer platform.
We never used MS Windows because we need not it. But we are sure that
the program is simply compilable at this obscured platform too.

\end
